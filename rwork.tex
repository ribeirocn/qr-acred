At its core\cite{lysyanskayaSignatureSchemesApplications2002}, anonymous credentials involve three main entities: issuers, holders, and verifiers. Issuers provide credentials to holders, who can present them to verifiers for access to services or acquire additional credentials. Holders must prove ownership of these credentials to verifiers, ensuring that they originate from trustworthy issuers. Privacy protection necessitates non-linkable presentations of credentials, even if all verifiers and issuers collaborate. Issuers should not be able to connect credentials issued to the same holder, but holders may present credentials from different issuers to demonstrate affiliation.

Beyond the core concept, subsequent systems for anonymous credentials introduced features such as selective disclosure (revealing specific attributes to verifiers)\cite{camenischEfficientAttributesAnonymous2012}, range proofs (e.g., age > 18), and proofs involving multiple attributes. Revoking credentials while maintaining anonymity is challenging, and there are limited solutions available\cite{laponAnalysisRevocationStrategies2011}. Pseudonyms \cite{camenischDesignImplementationIdemix2002} and limited-use credentials (restricting the number of times a credential can be used) are common features\cite{camenischEfficientAttributesAnonymous2012}.

Most proposals adhere to the original framework\cite{camenischEfficientSystemNontransferable2001}, where holders commit to a secret value and certain attributes, which are signed by issuers and proved to verifiers using a zero-knowledge proof. Efficiently combining a digital signature system and zero-knowledge proof is the primary challenge, with various solutions proposed, including RSA-based schemes, bilinear maps \cite{camenischSignatureSchemesAnonymous2004,camenischAnonymousAttestationUsing2016a}, and quantum-resistant approaches\cite{bootleFrameworkPracticalAnonymous2023,blazyEfficientImplementationPostQuantum2023,jeudyLatticeSignatureEfficient2022,laiLatticebasedCommitTransferrableSignatures2023}.

Different research groups, such as Jeudy et al. \cite{jeudyLatticeSignatureEfficient2022}, Blazy et al. \cite{blazyEfficientImplementationPostQuantum2023}, and Lai et al. \cite{laiLatticebasedCommitTransferrableSignatures2023}, have developed their versions of quantum-resistant anonymous credentials with varying features and trade-offs. Jeudy et al. \cite{jeudyLatticeSignatureEfficient2022} focus on efficient oblivious signatures and proofs, but their proofs are relatively large ($639KB$). Blazy et al. use a group signature scheme and a lattice-based zero-knowledge framework, but lack selective disclosure and have large proofs ($3.7MB$). Lai et al. \cite{laiLatticebasedCommitTransferrableSignatures2023} build upon a new construction for commitment transferability to ensure unlinkability, but still have relatively large proofs of $500KB$.

Addressing the challenge of efficiently demonstrating knowledge of random oracle preimages within zero-knowledge frameworks remains a common concern. Some proposals opt for standard-model signature frameworks. Notably, Bootle et al. \cite{bootleFrameworkPracticalAnonymous2023} build a random-oracle independent signature system, albeit relying on unproven assumptions, resulting in small proof sizes ($133KB$ for $16$ attributes).

Future developments in quantum-resistant anonymous credential systems are expected, exploring variants such as zkSNARKs on lattices\cite{boschiniEfficientPostquantumSNARKs2020}, functional encryption\cite{bonehFunctionalEncryptionDefinitions2011}, and homomorphic signatures\cite{gorbunovLeveledFullyHomomorphic2015}.


