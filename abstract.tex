The growing emphasis on privacy, driven by the abundance of electronic data and advances in analytical and artificial intelligence tools, has been reinforced by recent legislation like GDPR and CCPA. Identity management systems (IMS) are at the heart of privacy concerns, facing challenges related to traceability and overdisclosure, both online and offline. Anonymous Credentials, introduced by D. Chaum in 1985, have gained prominence, but recent research is exploring new approaches, often based on lattice-based assumptions, to address quantum computing threats.

This proposal takes a unique approach by leveraging homomorphic encryption and verifiable decryption methods to protect attributes and demonstrate specific computations over signed attributes without full disclosure. It offers advantages in versatile attribute verification, reducing the need for additional credentials, and addressing revocation concerns while maintaining privacy.

The solution is built on three key primitives, including a verifiable decryption mechanism, a method to perform inequalities and private membership tests in homomorphic encryptions, and a variant of the GPV signature suitable for homomorphic verification.