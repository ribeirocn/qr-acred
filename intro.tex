The focus on privacy has increased due to the vast amount of electronic data and advances in analytical and artificial intelligence tools. Recent legislation such as GDPR (General Data Protection Regulation) \cite{GDPR2016a} and CCPA (California Consumer Privacy Act) \cite{bukatyCaliforniaConsumerPrivacy2019} emphasize the need for strict transmission of private information. Identity management systems (IMS) are at the centre of this issue, as they must prove identity traits without leaking unnecessary information. Online IMS can have traceability problems, revealing the frequency of identifications and services contacted by individuals. Offline IMS, based on signed credentials, may transmit more information than necessary and can also have traceability issues when colluding with services.

The concept of Anonymous Credentials was first proposed by D. Chaun in 1985\cite{chaumSecurityIdentificationTransaction1985}, but it was not until 2001 that the first fully functional solution was proposed\cite{camenischEfficientSystemNontransferable2001}. Since then, there have been many different proposals targeting additional features\cite{laponAnalysisRevocationStrategies2011,garmanDecentralizedAnonymousCredentials2013,camenischEfficientAttributesAnonymous2012,connollyImprovedConstructionsAnonymous2022}, efficiency\cite{camenischAnonymousAttestationUsing2016a}, standardization\cite{curranAnonCredsSpecification2023}, and concrete implementations\cite{lodderAnonymousCredentials2019}. Recently, there has been a new line of research\cite{bootleFrameworkPracticalAnonymous2023,blazyEfficientImplementationPostQuantum2023,boschiniEfficientPostquantumSNARKs2020,jeudyLatticeSignatureEfficient2022,laiLatticebasedCommitTransferrableSignatures2023,boschiniRelaxedLatticeBasedSignatures2018} that aims to change the cryptographic assumptions used in these solutions, as current assumptions are believed to become obsolete with the advent of quantum computing. Most of the new proposed solutions for quantum-safe cryptography are based on lattice-based assumptions\cite{FIPS203,FIPS204}, but different strategies exist, using: (i) lattice-based zero-knowledge proof systems\cite{lyubashevskyLatticeBasedZeroKnowledgeProofs2022}, (ii) zkSNARKs\cite{boschiniEfficientPostquantumSNARKs2020}, or as in the present proposal, (iii) homomorphic encryption to prove attributes without complete disclosure. Each solution should be studied, improved, and compared with others to overcome its challenges.



Although most existing solutions rely on concealing credential attributes within signed commitments, which holders then selectively demonstrate to verifiers using zero-knowledge proofs, our approach takes a different route. We employ homomorphic encryption to shield these attributes and employ verifiable decryption methods to demonstrate the result of specific computations over signed attributes to verifiers, without disclosing the attributes.

This approach offers a significant advantage: it extends beyond the conventional boundaries of attribute disclosure. For instance, it can validate statements such as "older than 18" without divulging the precise birthdate or necessitating the issuance of additional credentials. Additionally, it can verify attributes such as "address is from an EU member state" by privately inferring this information from the member state's name in the address. Moreover, our solution can verify functions across attributes in various credentials from different issuers without requiring intermediate issuers to issue new credentials, which reduces the need for extensive interaction within the overall system.

Revocation is of paramount relevance in any solution with long-lived credentials. However, revocation is notably absent in many proposals. Our solution capitalizes on the expressiveness of our attribute-proof system to build a simple revocation mechanism that still preserves privacy.

The drawback of our proposal is the size of the credentials' presentations. While the credentials themselves can be less than $5KB$\footnote{Assuming that the holder knows the attributes the issuer only needs to send the signature.}, the presentations of the credentials (sent from the holder to the verifier) vary with the number of attributes and the depth of the function to be proved, which for the depth required by our comparison primitives (Section \ref{sec:comparison}) is $\approx734~KB$.

Our solution is based on three primitives that may be of independent interest: (i) a special verifiable decryption primitive, first suggested in \cite{chillottiHomomorphicLWEBased2016}; (ii) a primitive for performing simple inequalities and private membership tests in homomorphic encryptions; and (iii) a variant of the GPV signature that can be verified homomorphically.
