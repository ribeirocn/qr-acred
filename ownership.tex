The proof of ownership of a credential is no different from the proof of any other attribute. The holder provides the issuer with a committed value, and the issuer includes it as any other attribute. To prove ownership, the holder opens the commitment but does it homomorphically, not revealing the opener or the commitment.

The commitment can be any commitment with a linear opening, such as, for instance, the Ajtai commitment \cite{ajtaiGeneratingHardInstances1996} adapted to the ring setting. In such a commitment scheme, one commits to a secret vector $\pv{s} \in \intring_{d,p}^l$ with a randomness $\pv{r} \in \intring_{d,p}^l$, where both $\|\pv{s}\|$ and $\|\pv{r}\|$ are small, for some public random values $\pv{a}\from \intring_{d,p}^l$ and $\pv{b}\from \intring_{d,p}^l$, by computing the commitment
\begin{equation}\label{eq:commit}
\begin{aligned}
    \bkw{Comm}(&\pv{s},\pv{r})\colon \intring_{d,p}^{2l} \to \intring_{d,p}\\
    &\inprod{\pv{a}}{\pv{s}}+\inprod{\pv{b}}{\pv{r}} \mapsto  \pc
\end{aligned} 
\end{equation}
and sending it to the issuer to be signed, together with the issuers' attributes (cf. section \ref{sec:solution}).

Opening the Ajtai commitment is a matter of revealing the committed value $\pv{s}$ and the randomness $\pv{r}$, and check 
\begin{subequations}\label{eq:commitopen}
\begin{align}
    \bkw{OpenCom}(&\pc,\pv{s},\pv{r})\colon \intring_{d,p}^{2l+1} \to \{0,1\} \nonumber\\
    &\inprod{\pv{a}}{\pv{s}}+\inprod{\pv{b}}{\pv{r}} \stackrel{?}{=} \pc \label{eq:commitopen_a}\\
    &\|\pv{s}\| \textrm{ and } \|\pv{r}\| \textrm{ are small} \label{eq:commitopen_b}
\end{align} 
\end{subequations}
Implementing the commitment opening homomorphically over the presentation ciphertext $\p{prt} = \enc{\p{m}} \in \intring_q$
\begin{equation}\label{eq:hcommitopen}
    \overline{\bkw{OpenCom}}(\p{prt})\colon \intring_{q} \to \enc{\{0,1\}}\\
\end{equation}
can be done efficiently if the presentation $\p{prt}$ is the encryption of the canonical embedding of the different elements of the presentation 
\begin{equation}\label{eq:presentation}
    \p{prt}=\enc{\{\vv{0},sn,\vv{\sigma}(\pc),\vv{\mu},\vv{\sigma}(\pv{\varsigma}),\vv{\sigma}(\pv{o})\}}
\end{equation}
where $\pc$ is the commitment, $\vv{\mu}=\{\mu_0,\ldots\mu_p\}$ is the set of attributes, $\pv{\varsigma}$ is the credential signature (cf. section \ref{sec:sig}), $sn$ is the credential serial number, and $\pv{o}=(\pv{s},\pv{r}) \in \intring_{d,p}^{2l}$ is the commitment opener. Using standard automorphism operations available in $\BGV$ and $\BFV$, the verifier can 
verifier extract $\enc{\pv{s}}$, $\enc{\pv{r}}$ and $\enc{\pc}$ from the presentation $\p{prt} =\enc{\p{m}}$ and verify  homomorphically \cref{eq:commitopen_a,eq:commitopen_b}.

The \cref{eq:commitopen_a} can be checked efficiently homomorphically by computing 
\begin{equation}\label{eq:commitverify}
\SUM_{2l,d}\left((\pv{a},\pv{b}) \odot  \enc{\pv{o}}\right) \ominus \enc{\pc} 
\end{equation}
The result will be an encryption of 0 if the commitment was correctly opened or something else otherwise.
The operators $\odot$ and $\ominus$ denote the element-wise homomorphic multiplication and subtraction, respectively, and $\SUM_{m, n}(\enc{\pv{a}})$ is a homomorphic function that takes $m$ groups of $n$ slots of each encrypted polynomial $\enc{\pa_k}$ and returns an encrypted polynomial $\enc{\pb_k}$ with the first $n$ slots having the group-wise homomorphic addition, and the remaining slots having zero (vide \cref{eq:sumslots}).
\begin{equation}\label{eq:sumslots}
\begin{aligned}
    &\enc{\pv{b}} = \SUM_{m, n}(\enc{\pv{a}})\\
    &\vv{\sigma}(\pb_k)_i  = 
    \begin{cases}
        \overline{\sum}_{j=0}^{m-1} \vv{\sigma}(\pa_k)_{i+j.n} & \text{if } 0\leq i < n\\
        0 & \text{otherwise}
    \end{cases}
\end{aligned}    
\end{equation}
With this approach, checking \cref{eq:commitopen_a} requires only one homomorphic multiplication, $1+\log m$ additions, and $\log m$ automorphisms.

To simplify the process of verifying \cref{eq:commitopen_b}, we restrict $\pv{s}$ and $\pv{r}$ to have $\|\vv{\sigma}(\pv{s})\|_{\infty}=\|\vv{\sigma}(\pv{r})\|_{\infty}=1$, by chosen two random vectors $\vv{s},\vv{r}\from \{-1,0,1\}^{l.d}$ and canonical embedding them in $\pv{s}$ and $\pv{r}$, respectively. The verifier leverages the lemma \ref{lm:sign} to check that every slot of the opener $\vv{\sigma}(\p{o})_i \in \{-1,0,1\}$, and
the norm inequalities\cite{damgardMultipartyComputationSomewhat2012} to ensure that the norm
\[\|\pv{o}\|\leq\sqrt{d.(l+1)}\|\p{o}\|_{\infty}\leq\sqrt{d.(l+1)}\|\vv{\sigma}(\p{o})\|_{\infty}\leq\sqrt{d.(l+1)}\]

The commitment is statistically hiding and computationally binding, provided that $RSIS_{d,p,l,d(l+1)}$ is hard. From \cref{eq:rsisbound} we must set $d,l\geq\log p$ and $p$ so that the bound 
\[B=\sqrt{d.(l+1)}<\textsf{min}(p,2^{\sqrt{4d(l+1)\log{p}\log{\delta}}})\]

For reasons that will be evident in the next section, the plaintext modulus $p$ can be set to a prime number $p\approx O(2^{16})$. Setting $l$ to the minimum admissible value $l=16$, we have $d\ge 1$ for a hermit value $\delta=1.0043$. Choosing $d=4$ ensures a large enough commitment ($2^{64}$) that is not easily guessable. Therefore, the commitment takes only four slots in the credential, and the opening $2.4.16=128$ slots in the encrypted presentation message $\enc{\vv{\sigma}(m)}$. 

As usual, the commitment $\pc$ is the holder's identifier on the issuer. The holder may be known by different commitments $\pc'=\inprod{\pv{a}}{\pv{s}}+\inprod{\pv{b}}{\pv{r}'}$ to the same secret $\pv{s}$, in different issuers, by using different randomizations $\pv{r}'$. In this context, proving the ownership of two credentials from different issuers requires only extending the opener $\pv{o}=(\pv{s},\pv{r},\pv{r}')$ to include both randomizations.
